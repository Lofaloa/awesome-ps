\section{Module de sélection des processus}
\subsection{Vue d'ensemble}
L'objectif de ce module est de lister les processus actifs du système. Pour mettre cela en place, nous utilisons le système de fichier virtuel /proc. Il contient, à sa racine, un dossier pour chacun des processus actifs. Le fonctionnement de ce module est donc très simple. Il parcours la racine de /proc pour y sélectionner les dossiers qui contiennent les informations d'un processus. Ce type de dossier est nommé par le pid pour lequel il garde des informations (seul critère de sélection). Lors de la sélection, nous rajoutons des critères basés sur les informations du processus identifiés. Ces critères sont spécifiés par l'utilisateur.

Les critères de sélection disponibles sont :

\begin{itemize}
\item Sur base de l'identifiant du processus ;
\item Sur base de l’état du processus ;
\item Sur base de l’utilisateur du processus (identifiant réel de l'utilisateur) ;
\item Sur base de l’utilisateur et du shell courant (comme la commande “ps” 
fournie par Linux, il s'agit du comportement par défaut) ;
\end{itemize}

\subsection{Sur base de l'état du processus} 
Filtre où seul les processus dans un état donné par l’utilisateur seront donnés. Pour cela nous utilisons la fonction “processes\_filter.matchesState()” cette fonction va scanner le fichier à l’aide de la fonction "scanStatFile()” pour en extraire la lettre correspondant à l’état courant du process et la comparer avec l’entrée de l’utilisateur.

Pour cela nous utilisons également une fonction stateValueToCharacter() qui se charge de transformer l'entrée utilisateur en quelque chose de lisible par la fonction matchesState().

\begin{lstlisting}[frame=single, language=c]
static bool matchesState(int pid, char* state)
{
    char stateChar = stateValueToCharacter(state);
    process information;
    if (scanStatFile(pid, &information) == 0)
    {
        return stateChar == information.state;
    }
    return false;
}
\end{lstlisting}

\begin{itemize}
\item Conversion de l'entrée utilisateur via stateValueToCharacter()
\item Scan des informations du process via scanStatFile()
\item Comparaison du caractère qui définis l'état du process et l'entrée utilisateur convertie
\end{itemize}


\subsection{Sur base de l’utilisateur du process}
Filtre où seul les processus dont l’identidiant réel de l'utilisateur correspond au nom d'utilisateur spécifié seront sélectionnés. Pour cela nous utilisons la fonction “processes\_filter.matchesUserName()”. Qui va tout d’abord se charger de récupérer l’identifiant de l’utilisateur du processus avec getUserRealIdidentifier().

On va, ensuite, récupérer le nom de l'utilisateur du process avec la fonction findUserName() présente dans user\_informations.c qui va donc convertir un user id en nom. Cette dernière va récupérer l'id de l'utilisateur grâce à la fonction getpwuid() qui va se charger de renvoyer les informations d'un utilisateur dont on donne l'id en lisant dans le fichier /etc/passwd, là où sont stockés tous les utilisateurs.

\begin{lstlisting}[frame=single, language=c]
static bool matchesUserName(int pid, char* username)
{
    char processUsername[256];
    int processUserId = getUserRealIdentifier(pid);
    findUserName(processUserId, processUsername);
    return (strcmp(username, processUsername) == 0);
}
\end{lstlisting}



\subsection{Sur base de l’utilisateur et du terminal courant} 
Ici nous allons comme dans le ps de base, n'afficher que les process correspondants à l'utilisateur courant ainsi qu'au terminal courant. Pour cela nous utilisons matchCurrentUser() et matchesCurrentTerminal().

\begin{lstlisting}[frame=single, language=c]
static bool matchesCurrentUser(int pid)
{
    int userRealIdentifier = getUserRealIdentifier(pid);
    if (userRealIdentifier >= 0)
    {
        return getuid() == userRealIdentifier;
    }
    return false;
}
\end{lstlisting}

Dans matchesCurrentUser() : On compare l'utilisateur du process avec l'utilisateur du process courant. On utilise la fonction standard "getuid()" pour récupérer l'user id de l'utilisateur courant.


\begin{lstlisting}[frame=single, language=c]
static bool matchesCurrentTerminal(int pid)
{
    process information;
    if (scanStatFile(pid, &information) == 0)
    {
        return getppid() == information.ppid || information.pid == getppid();
    }
    return false;
}
\end{lstlisting}

Dans matchesCurrentTerminal() : On récupère les informations du process pour en trouver l'id du process père et le comparer à l'id du process courant. La fonction standard getppid() est utilisée pour récuperer l'id du père du process courant.

\subsection{Plusieurs options simultanées} 

Avec la fonction matchesOption() nous pouvons faire en sorte que le process soit choisi selon plusieurs critères au lieu d'un seul.