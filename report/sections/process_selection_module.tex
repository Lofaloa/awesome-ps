\section{Module de sélection des processus}
La sélection des processus se fait sur la base du process id, celui-ci est récupéré en parsant les dossiers de processus sous le dossier /proc. Le but étant de récupérer tous les dossiers ayant un nom alphanumérique pour les associer à un processus et récupérer leurs données. 

La sélection des process peut se faire de trois manières différentes par l’utilisateur :

\begin{itemize}
\item Sur base de l’état du process
\item Sur base de l’utilisateur du process
\item Sur base de l’utilisateur et du terminal courant (comme la commande “ps” 
fournie par linux)
\end{itemize}

\subsection{Sur base de l'état du process} 
Filtrage où seul les process dans un état spécifique donné par l’utilisateur seront donnés. Pour cela nous utilisons la fonction “process\_selector.matchStatus()” cette fonction va lire les informations contenues dans le fichier “/proc/[pid]/status” à l’aide de la fonction “scanStatusInformation()” pour en extraire la lettre correspondant à l’état courant du process et la comparer avec l’entrée de l’utilisateur.

\subsection{Sur base de l’utilisateur du process}
Filtrage où seul les process dont l’userId correspond à l’user donné par l’utilisateur seront donnés. Pour cela nous utilisons la fonction “process\_selector.mathUser()”. Qui va tout d’abord se charger de récupérer l’id de l’utilisateur du process id passé en paramètre grâce à la fonction “findProcessUserId()” puis ensuite de le convertir en nom d’utilisateur grâce à la fonction “findUserName()” pour enfin comparer l’entrée de l’utilisateur et le nom de l’utilisateur du process.

\subsection{Sur base de l’utilisateur et du terminal courant} 
La fonction va comme dans un ps de base trier les process en regardant si leur userId correspond au process courant et également faire un tri sur base du terminal en comparant le terminal du process (TTY) au terminal courant via la fonction “ttyname()”. 