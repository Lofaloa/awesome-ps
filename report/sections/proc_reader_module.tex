\section{Module de lecture des données}
Dans cette section, nous commençons par décrire brièvement le système de fichiers monté sur /proc. Ensuite, nous exposons les différentes étapes de la lecture. Nous expliquons également comment nous utilisons ces données et comment nous les interprétons.

\subsection{Lecture du fichier /proc/[pid]/stat}
\subsubsection{Ce que nous apprend la documentation}
Le fichier contient les informations qui concernent l’état d'un processus. Si l'on consulte son contenu, on y trouve une suite de données peu lisible.
Pour les comprendre et les utiliser, les pages de manuel (man 5 proc) nous explique que les données y sont rangées dans l'ordre séparées par un espace. Pour en extraire ses données, la documentation nous recommande d'utiliser la fonction scanf(3) de la librairie standard. La documentation spécifie un format pour chaque donnée. Par exemple, state est un caractère (%c).

\subsubsection{Implémentation de la lecture}
Nous commençons par écrire une structure de données capable de faire le lien entre le fichier et notre programme. Notre objectif est de récupérer et d’utiliser les valeurs qui y sont contenues.
Nous avons d’abord besoin de savoir par quel type de donnée sont représentées les valeurs. Pour le savoir, nous nous basons sur les formats spécifiés par la documentation. C’est très simple. Voici quelques exemples,

\begin{itemize}
\item Le champ "comm (\%s)" est représenté dans un buffer de 256 caractères dans la structure (status\_information.comm).
\item Le champ "state (\%c)" est représenté par un caractère (status\_information.state).
\item Le champ "minflt (\%lu)" est représenté par un long entier non signé (long unsigned).
\end{itemize}

La documentation nous indique que l’on peut utiliser la fonction standard scanf(3) pour le scanner. Par conséquent, le principe est d’utiliser cette fonction pour scanner le contenu et notre structure pour recevoir les valeurs scannées. Nous écrivons une fonction qui se chargera de faire cela pour nous. On veut pouvoir lui fournir le pid d’un processus et obtenir les informations relatives à son état.

\subsection{Lecture des données utilisateur}
Pour trouver l’utilisateur du process ainsi que son nom, nous créons 2 fonctions dans le fichier "user\_information.c" : 

\begin{itemize}
\item findProcessUserId(int pid) : Se charge de trouver l’id linux de l’utilisateur dont l’id du process a été passé en paramètres. Cette fonction va se charger de lire le fichier “/proc/[pid]/status” et d’en extraire l’user id.

\item findUserName(int userId) : Elle va se charger de convertir un id d’utilisateur en nom en chaîne de caractères correspondant au nom de l’utilisateur. Cette fonction va donc dans le fichier /etc/passwd lire l’entrée utilisateur correspondant à l’id donné en paramètre.
\end{itemize}