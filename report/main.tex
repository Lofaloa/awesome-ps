% \documentclass[french, a4, 12pt, titlepage]{article}
\documentclass[french, a4, 12pt]{article} % si la table des matiï¿œes est petite
\usepackage[utf8]{inputenc}
\usepackage[francais]{babel}

% Modification des marges ------------------------------
\oddsidemargin -4mm 	% Marge de gauche -4mm
\textwidth 17cm 	% Largeur de gauche = 17cm
\textheight 22cm 	% Hauteur du texte = 22cm
\parindent 0cm		% Pas d'indentation de paragraphe

% Packages --------------------------------------------
\usepackage[T1]{fontenc}
\usepackage{graphicx,color}
\usepackage{epsfig}
\usepackage{fancyhdr}
\usepackage{textcomp}
\usepackage{listings}		% pour incorporer des sources
\usepackage[francais]{layout}	% pour obtenir le layout
\usepackage{fullpage}		% pour obtenir le layout
\usepackage{makeidx}		% pour créer une table d'index

\setlength{\headheight}{15pt}
\setlength{\headsep}{0.2in}
\pagestyle{fancy}

\begin{document}

% Titres sur  chaque page -----------------------------
\lhead{AwesomePs - Rapport de laboratoire} % Haut gauche
%\chead{Haut centre}
% \rhead{}
\lfoot{Cours de SYSG5 (2019 - 2020)}
\cfoot{ } % Bas centre - obligatoire !
\rfoot{\thepage} % Bas droite
\renewcommand{\headrulewidth}{0.4pt} 
\renewcommand{\footrulewidth}{0.4pt} 

% Numérotation et table des matières -----------------
\setcounter{tocdepth}{1}    % fixe la profondeur de la table des matières
\setcounter{secnumdepth}{5} % fixe la profondeur de la numérotation des sections et paragraph

% Page de garde --------------------------------------
\title{\emph{\textbf{Rapport de laboratoire}}\\Utiliser le système de fichier /proc}
\author{Alexandre Baudot et Logan Farci}
\date{22 novembre 2019}
\maketitle
\newpage

\tableofcontents

% Inclusion des textes
\newpage
\section{Introduction}
L’objectif de notre projet est d’implémenter un outil en ligne de commande qui permet d’afficher des informations relatives aux processus actifs. Ces informations sont disponibles dans le système de fichiers monté sur le répertoire /proc. 
L’outil est semblable à la commande ps. Mais il se différencie en proposant des fonctionnalités plus simples.  La sélection des processus est basique. Elle permet à l’utilisateur de filtrer sur base de l’état (running, sleeping, waiting,...), de l’utilisateur ou du terminal courant. Quant à la sélection des données, elle se fait sur base de “sujets”. On retrouve les sujets suivant : la pagination, la vie d’un processus et la gestion des fichiers. Les sujets sont détaillés dans le fichier README.md du projet.\\

Dans ce document, nous expliquons comment nous avons implémenté cet outil. Nous allons parcourir le projet pour comprendre son fonctionnement dans le détail. Il se divise en trois modules, nous avons dédié une section à chacun d’eux :

\begin{itemize}
\item Le module de sélection des processus : il se charge de filtrer les processus du système d’exploitation sur base de critères. Il fournit une liste de numéros d’identification de processus (pid).
\item Le module de lecture des données : il se charge de lire le système de fichier monté sur /proc pour un processus spécifique.
\item Le module d’interaction avec l’utilisateur : il se charge de lire les paramètres fournis par l’utilisateur et d’afficher les données désirées pour les processus sélectionnés.
\end{itemize}

D’abord, nous décrivons le fonctionnement du module de sélection des processus. Il s’agit, d’une part, d’analyser le code qui permet de rechercher l’ensemble des numéros  d’identification de processus (pid) disponibles et, d’autre part, de comprendre comment nous les filtrons.\\
Ensuite, nous mettons en évidence les différentes étapes de la lecture du système de fichiers /proc. Nous parlons du code qui lit les informations intéressantes et disponibles. Nous expliquons également comment nous utilisons ces données et de leur interprétation.\\
Enfin, nous exposons les différents mécanismes utiliser pour gérer l’interaction avec l’utilisateur. Cela englobe l’affichage des données sur la sortie standard et la gestion des paramètres entrés par l’utilisateur à l’exécution du programme.


% \printindex			% Impression de la table des index
\end{document}
