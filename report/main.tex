\documentclass[french, a4, 12pt]{article}
\usepackage[utf8]{inputenc}
\usepackage[francais]{babel}

% Modification des marges ------------------------------
\oddsidemargin -4mm 	% Marge de gauche -4mm
\textwidth 17cm 	% Largeur de gauche = 17cm
\textheight 22cm 	% Hauteur du texte = 22cm
\parindent 0cm		% Pas d'indentation de paragraphe

% Packages --------------------------------------------
\usepackage[T1]{fontenc}
\usepackage{graphicx,color}
\usepackage{epsfig}
\usepackage{fancyhdr}
\usepackage{textcomp}
\usepackage{listings}		% pour incorporer des sources
\usepackage[francais]{layout}	% pour obtenir le layout
\usepackage{fullpage}		% pour obtenir le layout
\usepackage{makeidx}		% pour créer une table d'index

\setlength{\headheight}{15pt}
\setlength{\headsep}{0.2in}
\setlength{\parskip}{\baselineskip}
\setlength{\parindent}{0pt}
\pagestyle{fancy}

\begin{document}

% Titres sur  chaque page -----------------------------
\lhead{AwesomePs - Rapport de laboratoire} % Haut gauche
%\chead{Haut centre}
% \rhead{}
\lfoot{Cours de SYSG5 (2019 - 2020)}
\cfoot{ } % Bas centre - obligatoire !
\rfoot{\thepage} % Bas droite
\renewcommand{\headrulewidth}{0.4pt} 
\renewcommand{\footrulewidth}{0.4pt} 

% Numérotation et table des matières -----------------
\setcounter{tocdepth}{5}    % fixe la profondeur de la table des matières
\setcounter{secnumdepth}{5} % fixe la profondeur de la numérotation des sections et paragraph

% Page de garde --------------------------------------
\title{\emph{\textbf{Rapport de laboratoire}}\\Utiliser le système de fichiers /proc}
\author{Alexandre Baudot et Logan Farci}
\date{22 novembre 2019}
\maketitle
\newpage

\tableofcontents

% Inclusion des textes
\newpage
\section{Introduction}
L’objectif de notre projet est d’implémenter un outil en ligne de commande qui permet d’afficher des informations relatives aux processus actifs. Ces informations sont disponibles dans le système de fichiers monté sur le répertoire /proc.

L’outil est semblable à la commande ps. Mais il se différencie en proposant des fonctionnalités plus simples.  La sélection des processus est basique. Elle permet à l’utilisateur de filtrer sur base de l’état (running, sleeping, waiting,...), de l’utilisateur ou du terminal courant. Quant à la sélection des données, elle se fait sur base de “sujets”. On retrouve les sujets suivant : la pagination, la vie d’un processus et la gestion des fichiers. Les sujets sont détaillés dans le fichier README.md du projet.

Dans ce document, nous expliquons comment nous avons implémenté cet outil. Nous allons parcourir le projet pour comprendre son fonctionnement dans le détail. Il se divise en trois modules, nous avons dédié une section à chacun d’eux :

\begin{itemize}
\item Le module de sélection des processus : il se charge de filtrer les processus du système d’exploitation sur base de critères. Il fournit une liste de numéros d’identification de processus (pid).
\item Le module de lecture des données : il se charge de lire le système de fichier monté sur /proc pour un processus spécifique.
\item Le module d’interaction avec l’utilisateur : il se charge de lire les paramètres fournis par l’utilisateur et d’afficher les données désirées pour les processus sélectionnés.
\end{itemize}

D’abord, nous décrivons le fonctionnement du module de sélection des processus. Il s’agit, d’une part, d’analyser le code qui permet de rechercher l’ensemble des numéros  d’identification de processus (pid) disponibles et, d’autre part, de comprendre comment nous les filtrons.

Ensuite, nous mettons en évidence les différentes étapes de la lecture du système de fichiers /proc. Nous parlons du code qui lit les informations intéressantes et disponibles. Nous expliquons également comment nous utilisons ces données et de leur interprétation.

Enfin, nous exposons les différents mécanismes utiliser pour gérer l’interaction avec l’utilisateur. Cela englobe l’affichage des données sur la sortie standard et la gestion des paramètres entrés par l’utilisateur à l’exécution du programme.

\newpage
\section{Le système de fichiers /proc}
Cette section a pour but de donner, au lecteur, les clés pour comprendre le fonctionnement de notre outil. L'implémentation repose entièrement sur la lecture du système de fichiers /proc. Par conséquent, nous avons jugé qu'il était utile d'expliquer brièvement son fonctionnement et de décrire son contenu.

\subsection{Description}
/proc est un système de fichiers virtuel. Il permet d'interagir avec des resources propres au noyau du système d'exploitation. Il joue le rôle d'interface entre ces resources et un utilisateur (physique ou logiciel). Pour faciliter la lecture de /proc, il est présenté sous forme de système de fichiers. Ainsi, il peut être lu en utilisant les mêmes appels systèmes qu'une lecture d'un système de fichiers classique. Contrairement à cette deuxième catégorie, un système de fichiers virtuels n'est pas persisté. Il se trouve dans la mémoire R.A.M. lors de l'exécution du système. Il est créé dynamiquement au démarrage et monté sur le point /proc, il est ensuite tenu à jour par le noyau.

\subsection{Contenu}
Comme nous l'avons précisé au point précédent, les fichiers contenus dans /proc sont virtuels. Cela implique que leur taille est nulle. Ils sont créé dynamiquement et ils sont constamment mis à jour par le système. Nous catégorisons les fichiers fournis sur base de leur contenu.

D'une part, /proc fournit des fichiers qui nous renseignent sur l'état général du système, son hardware et ses périphériques. Par exemple, /proc/cpuinfo décrit le processeur, /proc/meminfo donne des statistiques sur l'utilisation de la mémoire et /proc/stat donne des statistiques sur le système. 

D'autre part, chaque processus actif est détaillé dans un répertoire nommé par son identifiant (pid). Par exemple, /proc/[pid]/cmdline contient la commande d'exécution complète, /proc/[pid]/exe est un lien symbolique vers l'exécutable du processus et /proc/[pid]/stat contient des informations sur l'état d'un processus.

Notons que dans le cadre de ce projet, l'utilisation de /proc se concentre sur la lecture des informations relatives à l'état d'un processus. Nous décrivons la lecture dans la section 4.
\newpage
\section{Module de sélection des processus}
\subsection{Vue d'ensemble}
L'objectif de ce module est de lister les processus actifs du système. Pour mettre cela en place, nous utilisons le système de fichier virtuel /proc. Il contient, à sa racine, un dossier pour chacun des processus actifs. Le fonctionnement de ce module est donc très simple. Il parcours la racine de /proc pour y sélectionner les dossiers qui contiennent les informations d'un processus. Ce type de dossier est nommé par le pid pour lequel il garde des informations (seul critère de sélection). Lors de la sélection, nous rajoutons des critères basés sur les informations du processus identifiés. Ces critères sont spécifiés par l'utilisateur.

Les critères de sélection disponibles sont :

\begin{itemize}
\item Sur base de l'identifiant du process ;
\item Sur base de l’état du process ;
\item Sur base de l’utilisateur du process ;
\item Sur base de l’utilisateur et du shell courant (comme la commande “ps” 
fournie par Linux) ;
\end{itemize}

\subsection{Sur base de l'état du process} 
Filtre où seul les process dans un état spécifique donné par l’utilisateur seront donnés. Pour cela nous utilisons la fonction “process\_selector.matchStatus()” cette fonction va lire les informations contenues dans le fichier “/proc/[pid]/status” à l’aide de la fonction “scanStatusInformation()” pour en extraire la lettre correspondant à l’état courant du process et la comparer avec l’entrée de l’utilisateur.

\subsection{Sur base de l’utilisateur du process}
Filtrage où seul les process dont l’userId correspond à l’user donné par l’utilisateur seront sélectionnés. Pour cela nous utilisons la fonction “process\_selector.mathUser()”. Qui va tout d’abord se charger de récupérer l’id de l’utilisateur du process id passé en paramètre grâce à la fonction “findProcessUserId()” puis ensuite de le convertir en nom d’utilisateur grâce à la fonction “findUserName()” pour enfin comparer l’entrée de l’utilisateur et le nom de l’utilisateur du process.

\subsection{Sur base de l’utilisateur et du terminal courant} 
La fonction va comme dans un ps de base trier les process en regardant si leur userId correspond au process courant et également faire un tri sur base du terminal en comparant le terminal du process (TTY) au terminal courant via la fonction “ttyname()”. 
\newpage
\section{Module de lecture des données}
Le module de lecture des données se charge de la lecture des informations diponibles dans le système de fichiers /proc. Notre volonté est de lui fournir un identifiant de processus pour qu'il puisse nous informer sur son état. Pour implémenter AwesomePs, nous avons réduit la lecture au nécessaire. Actuellement, le module permet de lire le contenu du fichier /proc/[pid]/stat et de /proc/[pid]/status. Dans le futur, nous pouvons imaginer élargir le champs de données disponibles.

\subsection{Lecture du fichier /proc/[pid]/stat}
\subsubsection{Ce que nous apprend la documentation}
Le fichier contient les informations qui concernent l’état d'un processus. Si l'on consulte son contenu, on y trouve une suite de données peu lisible.
Pour les comprendre et les utiliser, les pages de manuel (man 5 proc) nous explique que les données y sont rangées dans l'ordre séparées par un espace. Pour en extraire ses données, la documentation nous recommande d'utiliser la fonction scanf(3) de la librairie standard. La documentation spécifie un format pour chaque donnée. Par exemple, state est représenté par un caractère (\%c).

\subsubsection{Implémentation de la lecture}
Nous commençons par écrire une structure de données capable de faire le lien entre le fichier et notre programme. Notre objectif est de récupérer et d’utiliser les valeurs qui y sont contenues.
Nous avons d’abord besoin de savoir par quel type de donnée sont représentées les valeurs. Pour le savoir, nous nous basons sur les formats spécifiés par la documentation. C’est très simple. Voici quelques exemples,

\begin{itemize}
\item Le champ "comm (\%s)" est représenté dans un buffer de 256 caractères dans la structure (status\_information.comm).
\item Le champ "state (\%c)" est représenté par un caractère (status\_information.state).
\item Le champ "minflt (\%lu)" est représenté par un long entier non signé (long unsigned).
\end{itemize}

\pagebreak
Voici la définition complète de la structure. Vous pouvez retrouver le code complet dans le fichier sources/procfs\_reader/process.h.

\begin{lstlisting}[frame=single, language=c]
typedef struct process
{
    int pid;
    char comm[256];
    char state;
    // ... et autres ...
} process;
\end{lstlisting}

La documentation nous indique que l’on peut utiliser la fonction standard scanf(3) pour scanner le fichier. Le principe est donc simple. Nous utilisons cette fonction pour scanner le contenu du fichier et nous utilisons notre structure pour recevoir les valeurs scannées. 

Nous écrivons une fonction qui se chargera de faire cela pour nous. On veut pouvoir lui fournir le pid d’un processus et obtenir les informations relatives à son état dans la structure que l'on a définie. 

Notre premier objectif dans l'implémentation de la fonction est d'ouvrir le fichier stat associé au pid passé en paramètre. Pour l'ouvrir, nous utilisons la fonction standard fopen (3). Elle ouvre le fichier dont le chemin est spécifié en paramètre et y associe un flux de données. Nous avons donc besoin d'une chaîne de caractères qui représente son chemin.

Nous la construisons en utilisant la fonction sprintf. Elle a un comportement similaire à la fonction printf. Ici, la différence est que la chaîne n'est pas écrite sur la sortie standard mais dans un buffer dont l'adresse est passée en paramètre. Notre buffer se nomme path. Le chemin est disponible et il est, maintenant, possible d'appeler fopen pour ourvir notre fichier.

Après avoir ouvert le fichier, nous voulons lire son contenu sur base d'un format. Comme expliqué précédemment, les pages de manuels nous conseil d'utiliser une des fonctions de la famille scanf (3). Pour lire depuis le flux de données associé au fichier récemment ouvert, nous utilisons la fonction fscanf. Sa documentation nous apprend que la fonction lit le flux de données spécifié sur base du format fournit en paramètre. Nous connaissons les formats qui correspondent aux valeurs à lire grâce à la documentation (man 5 proc).

La fonction fscanf, sur base du format spécifié, interprète le flux de données et écrits les valeur interprétées dans les variables fournies. Nous lui faisons passer, dans l'ordre spécifié par la documentation, les différents champs de la structure que l'on a défini.

Voici le code final de la fonction. 
\begin{lstlisting}[frame=single, language=c]
int scanStatFile(int pid, process *information)
{
    if (information != NULL)
    {
        char path[BUFFER_SIZE];
        sprintf(path, "%s/%d/stat", PROCFS_ROOT, pid);
        FILE *fp = fopen(path, "r");
        if (fp != NULL)
        {
            fscanf(fp, process_FORMAT,
                   &(information->pid),
                   information->comm,
                   &(information->state),
                   // ... et autres ...
            );
            fclose(fp);
            return 0;
        }
    }
    return -1;
}
\end{lstlisting}

\subsection{Lecture des données utilisateur}
Au niveau des données utilisateur nous avons créé deux fonctions distinctes pour récupérer des des informations :

\begin{itemize}
\item getUserReadIdentifier(const int pid) : Trouver l'id de l'utilisateur correspondant au pid du process passé en paramètre.

\item findUserName(int userId) : Elle va se charger de convertir un id d’utilisateur en nom en chaîne de caractères correspondant au nom de l’utilisateur. Cette fonction va donc dans le fichier /etc/passwd lire l’entrée utilisateur correspondant à l’id donné en paramètre.
\end{itemize}

\subsubsection{getUserReadIdentifier() : Trouver l'id de l'utilisateur.}
Pour trouver l'id de l'utilisateur correspondant à un process id ici nous allons ouvrir le fichier /proc/[pid]/status et y lire l'entrée correspondant au RUID.

\begin{itemize}
\item L'ouverture du fichier se fait grâce à la fonction fonpenStatusFileOf(pid).

\item La récupération de l'id de l'utlisateur est réalisée par tokenisation des lignes dans le fichier status, l'id est récupéré lorsqu'on trouve l'entrée correspondante à "uid".
\end{itemize}

\begin{lstlisting}[frame=single, language=c]
int getUserRealIdentifier(const int pid)
{
    char line[BUFFER_SIZE];
    char *currentLineTokens[BUFFER_SIZE];
    FILE *status = fopenStatusFileOf(pid);
    while (fgets(line, sizeof(line), status))
    {
        tokenizeStatusFileLine(line, currentLineTokens);
        if (strcmp(currentLineTokens[0], USER_ID_KEY) == 0)
        {
            return parseStringToUnsigned(currentLineTokens[1]);
        }
    }
    fclose(status);
    return -1;
}
\end{lstlisting}

\subsubsection{findUserName() : Convertir l'id en nom d'utilisateur.}
La conversion est simple et utilise la fonction c getpwuid() qui nous permet d'avoir les informations correspondantes à un userId.

En récupérant le champ pw_name de la structure passwd on obtient le nom de l'utilisateur

\begin{lstlisting}[frame=single, language=c]
void findUserName(int userId, char *username)
{
    struct passwd *user = getpwuid(userId);
    if (user == NULL)
    {
        sprintf(username, "%d", userId);
    }
    else
    {
        sprintf(username, "%s", user->pw\_name);
    }
}
\end{lstlisting}




\newpage
\section{Module d'interaction avec l’utilisateur}
Dans cette section, nous exposons les différents mécanismes utilisés pour faciliter et gérer l’interaction avec l’utilisateur. Cela englobe l’affichage des données sur la sortie standard et la gestion des paramètres entrés par l’utilisateur à l’exécution du programme.

\subsection{Guide}
Le guide est un script bash qui a pour but de faciliter l'utilisation notre projet et de ses resources. Il permet à l'utilisateur de compiler le projet, de visualiser le rapport et de lancer une démonstration.

\subsection{Gestion des paramètres de programme}
\subsubsection{Vue d'ensemble}
AwesomePs permet à l'utilisateur de spécifier une série d'options à l'exécution du programme. Elles lui permettent d'indiquer des critères de sélection ou de choisir un ensemble de données à afficher (pour rappel un sujet regroupe des données relatives à un processus). Elles sont passées comme une série de clés et de valeurs lors de l'exécution.

Par exemple, un appel correct à l'exécutable pourrait ressembler à "./awesomeps topic=general state=sleeping". Ici, l'utilisateur demande à AwesomePs d'afficher les informations générales pour tous les processus qui ont l'état "sleeping". Vous pouvez retrouver l'ensemble des options disponibles dans le fichier README.md qui se trouve à la racine du projet.

\subsubsection{Implémentation de la gestion des paramètres}
Les options passées à l'exécution de AwesomePs doivent être interprétées avant de pouvoir être utilisées dans notre code. Pour commencer, nous définissons une simple structure. Elle nous permet de représenter une option passée en ligne de commande par un couple de deux chaînes de caractères. La première identifie la catégorie de l'option. La seconde identifie la valeur associée à la catégorie représentée.

Vous pouvez retrouver sa définition dans le fichier "sources/user\_interaction/awesomeps\_interaction.h".

\begin{lstlisting}[frame=single, language=c]
typedef struct awesomeps_option {
    char *key;
    char *value;
} awesomeps_option;
\end{lstlisting}

Les paramètres passés en ligne de commande sont directement interprétés. L'objectif est de convertir le tableau de chaines de caractères passé au programme en un tableau de structures. Cette étape est implémentée dans le fichier "sources/user\_interaction/awesomeps\_input.c".

Durant la lecture des paramètres de programme, nous interprétons les chaines de caractères en structures. Nous implémentons cette conversion dans la fonction qui suit.

\begin{lstlisting}[frame=single, language=c]
void setOptionFromString(char *str, awesomeps_option *option) {
    char *key = strtok(str, OPTION_SEPARATOR);
    char *value = strtok(NULL, OPTION_SEPARATOR);
    if (isValidKey(key)) {
        if (isValidValueForKey(key, value)) {
            option->key = key;
            option->value = value;
        } else {
            printf(
                "Option parsing error: %s isn't a valid value for %s.\n",
                value, key);
            exit(-1);
        }
    } else {
        printf("Option parsing error: %s is not a valid option\n", str);
        exit(-1);
    }
}
\end{lstlisting}

Cette fonction lit la chaine de caractères passée en paramètres et initialise une structure awesomeps\_option. Elle l'initialise avec les valeurs extraites de "str". Pour séparer la clé de sa valeur, nous utilisons la fonction standard strtok. Après la séparation, nous vérifions la validité de la clé et de la valeur avant d'initialiser la structure.

La clé doit être disponibles. Nous avons définit une liste clés que nous voulons mettre à disposition. La clé doit être l'une d'entre elles. Et la valeur doit être intègre. Nous tranchons sur base de la clé (chaque clé à un domaine prédéfini). Dans le cas la validité n'est pas assurée, la structure n'est pas initialisée et le programme sort avec un status d'erreur. Nous devons sortir puisque l'utilisateur doit être informé de son erreur.

Après la conversion chaque structures est ajoutée à un tableau. Le tableau serait ensuite utilisé par notre programme pour répondre au demande de l'utilisateur.

\subsection{Affichage des données}
Nous avons pris la décision d'organiser les données relatives aux processus sur base de sujets. Les sujets représentent un moyen simplifié de sélectionner les données d'un processus. Pour chacun d'eux, nous avons sélectionné une série de données que nous avons jugées pertinentes. Actuellement, l'utilisateur peut utiliser les sujets suivants: général, pagination et temps. Dans le futur, nous pouvons imaginer en ajouter beaucoup plus.

L'affichage est organisé sous forme de tableau. Chaque colonne représente une donnée. Chaque ligne représente un processus. Voici un exemple d'affichage.

\begin{lstlisting}[frame=single, language=c]
$ > ./awesomeps
+------------+------------+-------------+------------+
|PID         |COMMANDE    |STATE        |TERMINAL    |
+------------+------------+-------------+------------+
|8281        |(bash)      |Sleeping     |           2|
|8304        |(awesomeps) |Running      |           2|
+------------+------------+-------------+------------+
|PID         |COMMANDE    |STATE        |TERMINAL    |
+------------+------------+-------------+------------+

Current time: 2019-11-22 11:29:17

$ >
\end{lstlisting}

Cette exemple montre un affichage par défaut. Ici, seule les processus enfant du bash courant sont affichés. Le sujet par défaut est le général.

\subsubsection{Sujets disponibles et leur implémentation}
\begin{itemize}
\item Le sujet par défaut est appelé "general". Il s'agit d'un sujet qui a pour but de sélectionner les informations basiques d'un processus. Le pid, le nom de la commande, l'état et le terminal courant du processus sont affichés.

\item "paging" est un sujet relatif à la pagination d'un processus. Il affiche le pid, la commande, le nombre de défaut de page majeur et mineur.

\item "time" est un sujet relatif à la durée d'exécution d'un processus. Il affiche le pid, la commande, le temps passé en mode user et mode système.
\end{itemize}
\end{document}
