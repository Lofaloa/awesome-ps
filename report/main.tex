\documentclass[french, a4, 12pt]{article}
\usepackage[utf8]{inputenc}
\usepackage[francais]{babel}

% Modification des marges ------------------------------
\oddsidemargin -4mm 	% Marge de gauche -4mm
\textwidth 17cm 	% Largeur de gauche = 17cm
\textheight 22cm 	% Hauteur du texte = 22cm
\parindent 0cm		% Pas d'indentation de paragraphe

% Packages --------------------------------------------
\usepackage[T1]{fontenc}
\usepackage{graphicx,color}
\usepackage{epsfig}
\usepackage{fancyhdr}
\usepackage{textcomp}
\usepackage{listings}		% pour incorporer des sources
\usepackage[francais]{layout}	% pour obtenir le layout
\usepackage{fullpage}		% pour obtenir le layout
\usepackage{makeidx}		% pour créer une table d'index

\setlength{\headheight}{15pt}
\setlength{\headsep}{0.2in}
\setlength{\parskip}{\baselineskip}
\setlength{\parindent}{0pt}
\pagestyle{fancy}

\begin{document}

% Titres sur  chaque page -----------------------------
\lhead{AwesomePs - Rapport de laboratoire} % Haut gauche
%\chead{Haut centre}
% \rhead{}
\lfoot{Cours de SYSG5 (2019 - 2020)}
\cfoot{ } % Bas centre - obligatoire !
\rfoot{\thepage} % Bas droite
\renewcommand{\headrulewidth}{0.4pt} 
\renewcommand{\footrulewidth}{0.4pt} 

% Numérotation et table des matières -----------------
\setcounter{tocdepth}{5}    % fixe la profondeur de la table des matières
\setcounter{secnumdepth}{5} % fixe la profondeur de la numérotation des sections et paragraph

% Page de garde --------------------------------------
\title{\emph{\textbf{Rapport de laboratoire}}\\Utiliser le système de fichier /proc}
\author{Alexandre Baudot et Logan Farci}
\date{22 novembre 2019}
\maketitle
\newpage

\tableofcontents

% Inclusion des textes
\newpage
\section{Introduction}
Ici vient l'Introduction

\end{document}
