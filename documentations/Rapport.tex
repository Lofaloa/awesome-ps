%\documentclass[french, a4, 12pt, titlepage]{article}
\documentclass[french, a4, 10pt]{article} % si la table des matiï¿œes est petite
\usepackage[utf8]{inputenc}
%\usepackage[latin1]{inputenc}
\usepackage[francais]{babel}
% Modification des marges ------------------------------
\oddsidemargin -4mm 	% Marge de gauche -4mm
\textwidth 17cm 	% Largeur de gauche = 17cm
\textheight 22cm 	% Hauteur du texte = 22cm
\parindent 0cm		% Pas d'indentation de paragraphe
% -----------------------------------------------------
\usepackage[T1]{fontenc}
\usepackage{graphicx,color, caption2}
\usepackage{epsfig}
\usepackage{fancyhdr}
%\usepackage{fancyvrb}
\usepackage{textcomp}
\pagestyle{fancy}
\usepackage{listings}		% pour incorporer des sources
\usepackage[francais]{layout}	% pour obtenir le layout
\usepackage{fullpage}		% pour obtenir le layout
\usepackage{makeidx}		% pour créer une table d'index
%\usepackage{shadow}		% pour faire des encadrements
\begin{document}
% Titres sur  chaque page -----------------------------
\lhead{ } % Haut gauche
%\chead{Haut centre}
\rhead{Mon premier rapport en Latex}
\lfoot{\copyright{Logan Farci}}
\cfoot{ } % Bas centre - obligatoire !
\rfoot{\thepage} % Bas droite
\renewcommand{\headrulewidth}{0.4pt} 
\renewcommand{\footrulewidth}{0.4pt} 
% Numérotation et table des matières -----------------
\setcounter{tocdepth}{1}    % fixe la profondeur de la table des matières
\setcounter{secnumdepth}{5} % fixe la profondeur de la numérotation des sections et paragraph
% Page de garde --------------------------------------
\title{\emph{\textbf{Projet du cours de SYSG5}}\\Awesome PS}
\author{Logan Farci}
\date{20 septembre 2019}
\maketitle
\tableofcontents
\section{Introduction}
\subsection{Objectifs du projet}
% Inclusion des textes
\section{Introduction}

%-----------------------------------------------------------------------------------
\subsection {Objectifs}

%-----------------------------------------------------------------------------------
\section{Conclusions}
%-----------------------------------------------------------------------------------
\section{Références}
\begin{enumerate}
\item http://www.grappa.univ-lille3.fr/FAQ-LaTeX/ 
\item http://tex.loria.fr/
\item http://tex.loria.fr/english/packages.html
\end{enumerate}
%-----------------------------------------------------------------------------------

\printindex			% Impression de la table des index
\end{document}
