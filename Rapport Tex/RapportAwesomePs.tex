\documentclass[french,10pt,A4]{report}
%\usepackage[latin1]{inputenc}
\usepackage[utf8]{inputenc}
% Modification des marges ------------------------------
\oddsidemargin -4mm % Decreases the left margin by 4mm
\textwidth 17cm %Sets text width across page = 17cm
\textheight 22cm %Sets text height up and down = 22cm
% -----------------------------------------------------
\usepackage[T1]{fontenc}
\usepackage{graphicx,color, caption2}
\usepackage{epsfig}
\usepackage{fancyhdr}
\pagestyle{fancy}
\usepackage{listings}
\usepackage{color}
\usepackage{makeidx}

% Valeurs par défaut le lstset
\lstset{language={},%C,Assembleur, TeX, tcl, basic, cobol, fortran, logo, make, pascal, perl, prolog, {}
	literate={â}{{\^a}}1 {ê}{{\^e}}1 {î}{{\^i}}1 {ô}{{\^o}}1 {û}{{\^u}}1
		 {ä}{{\"a}}1 {ë}{{\"e}}1 {ï}{{\"i}}1 {ö}{{\"o}}1 {ü}{{\"u}}1
		 {à}{{\`a}}1 {é}{{\'e}}1 {è}{{\`e}}1 {ù}{{\`u}}1 
		 {Â}{{\^A}}1 {Ê}{{\^E}}1 {Î}{{\^I}}1 {Ô}{{\^O}}1 {Û}{{\^U}}1
		 {Ä}{{\"A}}1 {Ë}{{\"E}}1 {Ï}{{\"I}}1 {Ö}{{\"O}}1 {Ü}{{\"U}}1
		 {À}{{\`A}}1 {É}{{\'E}}1 {È}{{\`E}}1 {Ù}{{\`U}}1,
	commentstyle=\scriptsize\ttfamily\slshape, % style des commentaires
	basicstyle=\scriptsize\ttfamily, % style par défaut
	keywordstyle=\scriptsize\rmfamily\bfseries,% style des mots-clés
	backgroundcolor=\color[rgb]{.95,.95,.95}, % couleur de fond : gris clair
	framerule=0.5pt,% Taille des bords
	frame=trbl,% Style du cadre
	frameround=tttt, % Bords arrondis 
	tabsize=3, % Taille des tabulations
%	extendedchars=\true, % Incompatible avec utf8 et literate
	inputencoding=utf8,
	showspaces=false, % Ne montre pas les espaces 
	showstringspaces=false, % Ne montre pas les espaces entre ''
	xrightmargin=-1cm, % Retrait gauche 
	xleftmargin=-1cm, % Retrait droit
	escapechar=°}  % Caractère d'échappement, permet des commandes latex dans la source
% -----------------------------------------------------
%\makeindex
\begin{document}
\lhead{Exercice test}
\rhead{Page \thepage}
\cfoot{ }
\renewcommand{\footrulewidth}{0.4pt}

\setlength{\parindent}{0pt} % pas d'indentation

\lstset{frame=trBL}

\setcounter{tocdepth}{1}	% limiter les nivaux de table des matières
\setcounter{secnumdepth}{5}	% La numérotation des sections au maximum

\newcommand{\titre}{Titre du sujet}	% la variable contenant le titre du sujet

\thispagestyle{empty}

\title{\emph{Rapport SYSG5\\\textbf{Implémentation de PS}}}
\author{Logan Farci - Alexandre Baudot}
\date{Novembre 2019}
\maketitle
\tableofcontents

\newpage
%
\chapter{Chapitre 1}
	\lstset{language=c}
\renewcommand{\titre}{\textcolor{blue}{ Titre }}

\lhead{ \titre }
\section{{\titre} }

\begin{tabular}{|l|l|}
\hline
Titre : 	& \titre \\\hline
Support : 	& Je ne sais pas\\\hline
Date :		& 08/2019 \\\hline
\end{tabular}

\subsection{Énoncé}

Enoncé ici

\subsection{sous section}

Code ici 

\lstinputlisting{sources/SOURCES/code.c}

\subsection{sous section 2}

\begin{itemize}
\item item 1
\item item 2 ...
\end{itemize}

\subsection{sous section 3}

Test.\\
Fin.
\newpage
 
\end{document}
